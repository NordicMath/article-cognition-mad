
\section{The MAD framework}



\subsection{Overview}

The Angelsen square is a summary of certain classes of machine types. In one version, the nodes are (reading from top left, row by row): E x C, E, Ob(D); $C^2$, C, Mor(D); BOOL, REPR, FORMAL STATEMENTS.

A smaller version is given by $C^2$, C; BOOL, REPR


\subsection{Relevant literature}

List of relevant literature:

\begin{itemize}
\item Polya: How to solve it
\item Polya: Mathematics and plausible reasoning (2 volumes)
\item Polya: Mathematical Discovery: On Understanding, Learning and Teaching Problem Solving (originally 2 volumes)
\item (Polya: Mathematical methods in science)
\item Lockhart: A Mathematician's Lament \url{https://www.maa.org/external_archive/devlin/LockhartsLament.pdf}
\item Lautman
\item Ganesalingam
\item Rabe
\item Something by Manin?
\item Something by Mazur?
\item Corfield


\end{itemize}

References for how the brain works:

\begin{itemize}
\item McGilchrist: The Master and his Emissary

\end{itemize}

Works of pure mathematics for illustration purposes:
\begin{itemize}
\item Scholze's ICM article

\end{itemize}

\subsection{Machines}

Should we use punction and punctor?

\subsection{Analogies}

Examples of analogies:


\begin{itemize}
\item Integers vs polynomials, mention binary ops, factorization, gcd and lcm, algorithms for these, further analogies for example with factorization theorems for matrices (LU etc).
\item Real numbers vs integer sequences (with finiteness properties, operations, etc)
\item Local zeta functions vs global zeta functions
\item Inclusion of Multrat into Mult vs inclusion of $\mathbb{Q}$ into $\mathbb{R}$
\item Motives vs integers (or rational numbers)
\item Motives vs representations of a finite group
\end{itemize}




\subsection{Dialectics}

List of things to include:

\begin{itemize}
\item Everything from the Angelsen square. One version is mentioned above, a slightly different version is in an image files under notes in this repository.
\item What algebraic structure? What metric/topological structure? Ask about examples? About formal definitions? See further ideas in old Xtreme post?
\item A list from discussion at Bro: Representations, Examples, Unary Operations, Binary operations, Properties, Relations (internal), Invariants, Coinvariants, Actions, Relations (external), Pairings, Object interpretations, Morphism interpretations. Definitions, Theorems, Conjectures. Generalizations, Specializations. More categorical concepts like comma cats etc? Structure theorems, Decomposition theorems, Finiteness properties?
\end{itemize}

\subsubsection{Concept formation}

We want automatic generation of concepts to be part of the dialectics.

Examples:

\begin{itemize}
\item Whenever two types A and B are given (possibly equal), we can consider the type which is machines from A to B. For example, since real numbers is a type, we automatically have a type for (partially defined) functions from $\mathbb{R}$ to $\mathbb{R}$.
\item Set, Finite set, List, Finite list, Sequence??, Multiset, Hybrid set. I also wrote Monad (???).

\end{itemize}
