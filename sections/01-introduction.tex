
\section{Introduction}



\subsection{Goals}

Main goal: Achieve examples of deep and creative mathematical reasoning by a computer. By "deep" we mean that the reasoning should concern non-elementary mathematical objects and research-level statements about these objects. By "creative" we mean that the mathematical ideas generated are recognized by human experts as new and of significant value.

Auxiliary goals: Create a theoretical framework for future development of automated teaching tools for high-level mathematical reasoning. Build a phenomenological approach to the philosophy of mathematics, which addresses in constructive ways the problems identified by Zalamea in his book. Build a new structure for mathematical knowledge that can help students master the vast array of concepts in pure and applied mathematics. Understand the nature of non-formal/right-hemisphere mathematics.


\subsection{Philosophical notes}

"Phenomenology is a broad discipline and method of inquiry in philosophy, developed largely by the German philosophers Edmund Husserl and Martin Heidegger, which is based on the premise that reality consists of objects and events ("phenomena") as they are perceived or understood in the human consciousness, and not of anything independent of human consciousness."
From \url{https://www.philosophybasics.com/branch_phenomenology.html}

It seems reasonable to say that we are trying to develop a phenomenological understanding of Mathematics.

David Corfield (using the term "real mathematics") and Fernando Zalamea have both argued that current philosophy of mathematics to a large extent studies only a formal fragment of actual mathematics, and therefore has very little to do with mathematics as it is perceived and practiced by contemporary mathematicians. It is possible that our approach could be part of an effort to address these concerns.

I have a note mentioning "the MacLane question", but I'm note sure what this means.

\subsection{Our preliminary notes}

What is it that computers cannot currently do?
\begin{itemize}
\item Graphical calculus?? (See email????)

\item Emotional processing of mathematics.

\item Reasoning by analogy

\item Humans have both semantic and episodic memory, and computer memory is more like semantic memory.

\end{itemize}

Attempts to formulate the point of MAD:

\begin{itemize}
\item Mathematics consists of a formal layer and an informal layer. The problem of current foundations is that the informal layer is ignored. Illustrating example: The paper of Scholze. One cannot say that the analogies put forward are not part of mathematics.
\item Reviewing current AI mathematics, the conclusion is that computers today cannot really do deep and creative mathematical research. Our hypothesis is that the problem lies not in the nature of computers, but in our mental model of what mathematics is. If this hypothesis is true, real AI mathematics will only come about after a paradigm shift in the way we think about mathematics. The MAD framework is an attempt to formulate a new and better model. We illustrate the power of the framework by generating a number of analogies and conjectures of a highly nontrivial nature, culminating in a conjecture for elliptic curves of rank 2 that is a weak version of the BSD conjecture for such curves.

\end{itemize}

Motivating examples:
\begin{itemize}
\item Imagine another culture where 2x2 matrices are written in a different format, and maybe where linear algebra is done on 2-dimensional vector spaces only. Do they have the same linear as us (in 2 dimensions)??? From a formal point of view, yes, but from a phenomenological point of view, no.
\item Find examples of mathematical beauty and ask how to formalize the brain processes that correspond to experiencing beauty.
\end{itemize}

One could maybe say that M covers left-hemisphere maths, A covers right-hemisphere maths, and D covers the narrative and weaving of the mathematical universe.

One could maybe say that we replace axioms by cognition (as the foundation of mathematics).

We want to pay close attention to the role of cognition.

We want to accept mathematics that is not formalized, like some of the physics literature or some of the speculation on the field with one element.

I don't know where, but I saw some image which divided brain responsibilities by placing Logic, Analysis, Facts, Computation, Language and (hmm, does it say Linear?) in the LEFT, and Creativity, Imagination, Holistic, Intuition, Feeling, Visualization, Daydreaming in the RIGHT hemisphere.
